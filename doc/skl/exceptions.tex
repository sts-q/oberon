\chap{Exceptions}

\deliberate error
% This file is not used in the documentation.  Fault / Trap
% information needs to be rewritten when implemented.


exceptions are reaised and the CPU begins executing code at the
address stored in the exception control register.  If the value is
zero, then the CPU will halt, and only a reset can bring it back.

\section{Sequence of Events on Exception}

When an exception occurs, the following events happen atomically:

\begin{itemize}
\item \emph{Exception Cause} is put into \emph{Cause Register}.
\item \emph{Interrupt} State is put into \emph{Cause Register}.
\item \texttt{PC} is placed into \texttt{cr0}.
\item State in \texttt{cr2} is updated.
\item \texttt{cr1} is loaded into the \texttt{PC}.
\item Processor resumes execution at \texttt{PC}.
\item interrupts are disabled
\end{itemize}

\section{Sequence of Events on \texttt{ERET}}

\begin{itemize}
\item PC is loaded with \texttt{cr0}
\item Interrupts are set to the value contained in the \emph{Cause Register}.
\item Processor resumes execution at \texttt{PC}.
\end{itemize}


\section{Faults}

Faults are generated inside the \skl processor for actions such as
\emph{undefined instructions}.

Traps occur before the instruction is executed.  The return eip is the
instruction which caused the fault.

\subsection{Misaligned Memory Access}
\subsection{Misaligned Instruction Fetch}
\subsection{Breakpoint}
\subsection{Undefined Opcode}
\subsection{Divide by Zero}

\section{Interrupts}

Interrupts are generated external to the \skl processor and are
reported to the \skl processor via the interrupt pin.

Interrupts occur after an instruction is retired.  The return eip is
the next instruction.

