\newcommand{\mips}{MIPS\xspace}
\newcommand{\natsemi}{NS32000\xspace}
\newcommand{\skl}{\texttt{SKL}\xspace}

\makeatletter
\newcommand*{\bdiv}{% bdiv analog of bmod
  \nonscript\mskip-\medmuskip\mkern5mu%
  \mathbin{\operator@font div}\penalty900\mkern5mu%
  \nonscript\mskip-\medmuskip
}
\makeatother


\newcommand{\abs}[1]{\textrm{abs(}#1\textrm{)}}
\newcommand{\constant}{\textrm{C}\xspace}
\newcommand{\becomes}{\ensuremath{\leftarrow}}
\newcommand{\implies}{\Rightarrow}
\newlength{\veelen}
\settowidth{\veelen}{\ensuremath{\vee}}
\newcommand{\xor}{\makebox[\width]{\ensuremath{\vee\hspace{-\veelen}}\rule[-.5pt]{\veelen}{.5pt}}}
\newcommand{\asmreg}[1]{\texttt{R#1}\xspace}
\newcommand{\fasmreg}[1]{\texttt{F#1}\xspace}
\newcommand{\instreg}[1]{\ensuremath{\textrm{r}_{#1}}\xspace}
\newcommand{\regbank}[1]{#1}
\newcommand{\creg}[1]{\ensuremath{\textrm{CR}_{#1}}}
\newcommand{\sr}[1]{\instreg{#1}}
\newcommand{\fr}[1]{\regbank{#1}}
\newcommand{\dr}{\instreg{d}}
\newcommand{\const}{constant\xspace}
\newcommand{\pc}{\textrm{PC}\xspace}
\newcommand{\eip}{\textrm{EIP}\xspace}
\newcommand{\neip}{\textrm{next }\eip\xspace}
\newcommand{\xref}[1]{\S\ref{#1}}
\newcommand{\Figref}[1]{Figure~\ref{fig:#1}\xspace}
\newcommand{\figref}[1]{figure~\ref{fig:#1}\xspace}
\newcommand{\xrefinst}[1]{\texttt{#1}\xspace(\xref{inst:#1})\xspace}
\newenvironment{instruction}[1]{
  \section{\texttt{#1}}
  \index{\texttt{#1}}
  \vspace{-\baselineskip}\rule{\textwidth}{.5pt}
}{}


\newenvironment{xnote}{
  \begin{center}
    \begin{tabular}{|p{0.9\textwidth}|}
    \hline\\
    }
    {
    \\\\\hline
    \end{tabular}
    \end{center}
    }
% Instruction class encodings.
\newcommand{\gregtypeclass}{0}    % General Register class
   \newcommand{\aaddopc}{0}
   \newcommand{\asubopc}{1}
   \newcommand{\amulopc}{2}
   \newcommand{\adivopc}{3}
   \newcommand{\amodopc}{4}
   \newcommand{\acmpopc}{5}
   \newcommand{\aabsopc}{6}

\newcommand{\rtypeclass}{1}     % Integer Register Instructions
   \newcommand{\randopc}{0}
   \newcommand{\rashopc}{1}
   \newcommand{\rbitsetopc}{2}
   \newcommand{\rcmpsopc}{3}
   \newcommand{\rlshopc}{4}
   \newcommand{\rnoropc}{5}
   \newcommand{\roropc}{6}
   \newcommand{\rrotropc}{7}
   \newcommand{\rxoropc}{8}

\newcommand{\exttypeclass}{2}   % Sign extension.
  \newcommand{\extb}{0}         % Sign extend byte.
  \newcommand{\exth}{1}         % Sign extend halfword.

\newcommand{\crtypeclass}{3}
  \newcommand{\crlcr}{0}
  \newcommand{\crscr}{1}

\newcommand{\sregtypeclass}{4}  % System Register
  \newcommand{\sregdiopc}{0}    % This instruction is not implemented.
  \newcommand{\sregeiopc}{1}    % This instruction is not implemented.
  \newcommand{\sreglccopc}{2}

\newcommand{\mtypeclass}{5}     % These instructions are not implemented.
  \newcommand{\mbreakopc}{0}
  \newcommand{\mwaitopc}{1}
  \newcommand{\meretopc}{2}
  \newcommand{\mvmsvcopc}{3}

\newcommand{\jraltypeclass}{6}

\newcommand{\jumptypeclass}{7}
   \newcommand{\jjeqopc}{0}
   \newcommand{\jjneopc}{1}
   \newcommand{\jjltopc}{2}
   \newcommand{\jjgeopc}{3}
   \newcommand{\jjleopc}{4}
   \newcommand{\jjgtopc}{5}
   \newcommand{\jjltuopc}{6}
   \newcommand{\jjgeuopc}{7}
   \newcommand{\jjleuopc}{8}
   \newcommand{\jjgtuopc}{9}
   \newcommand{\jopc}{10}
   \newcommand{\jalopc}{11}

\newcommand{\regmemtypeclass}{8}
   \newcommand{\lbopc}{0}
   \newcommand{\lbuopc}{1}
   \newcommand{\ldopc}{2}
   \newcommand{\ldiopc}{3}
   \newcommand{\lfopc}{4}
   \newcommand{\lfiopc}{5}
   \newcommand{\lhopc}{6}
   \newcommand{\lhuopc}{7}
   \newcommand{\lwopc}{8}
   \newcommand{\lwiopc}{9}
   \newcommand{\sbopc}{10}
   \newcommand{\sdopc}{11}
   \newcommand{\sfopc}{12}
   \newcommand{\shopc}{13}
   \newcommand{\swopc}{14}
   \newcommand{\laopc}{15}

\newcommand{\bttypeclass}{9}     % Bit Test
   \newcommand{\btrropc}{0}       % bt   %r0, %r1, %rd
   \newcommand{\btcropc}{1}       % bt   C, %r1, %rd
   \newcommand{\btmopc}{2}        % btm  %r0, (%r1), %rd
   \newcommand{\btmiopc}{3}       % btmi C, (%r1), %rd
   \newcommand{\btmcopc}{4}       % btmc %r0, (%r1), %rd
   \newcommand{\btmciopc}{5}      % btmci C, (%r1), %rd
   \newcommand{\btmsopc}{6}       % btms %r0, (%r1), %rd
   \newcommand{\btmsiopc}{7}      % btmsi C, (%r1), %rd

\newcommand{\stktypeclass}{10}     % Stack
   \newcommand{\enteropc}{0}
   \newcommand{\leaveopc}{1}
   \newcommand{\pushopc}{2}
   \newcommand{\pushfopc}{3}
   \newcommand{\pushdopc}{4}
   \newcommand{\popopc}{5}
   \newcommand{\popfopc}{6}
   \newcommand{\popdopc}{7}

\newcommand{\ctypeclass}{11}    % Conditional set class
   \newcommand{\seqopc}{0}
   \newcommand{\sgeopc}{1}
   \newcommand{\sgeuopc}{2}
   \newcommand{\sgtopc}{3}
   \newcommand{\sgtuopc}{4}
   \newcommand{\sleopc}{5}
   \newcommand{\sleuopc}{6}
   \newcommand{\sltopc}{7}
   \newcommand{\sltuopc}{8}
   \newcommand{\sneopc}{9}

\newcommand{\systraptypeclass}{12} % Systrap instruction class
\newcommand{\trapnilopc}{0}        % trap if input is NIL

\newenvironment{InstructionInfo}[1]%
{\begin{list}{}{\renewcommand{\makelabel}[1]{\textsf{##1:}\hfil}%
      \settowidth{\labelwidth}{\textsf{#1:}}%
      \setlength{\leftmargin}{\labelwidth+\labelsep}}}%
{\end{list}}


\newsavebox{\stktypebox}
\newcommand{\stkbox}[2]{
  \begin{lrbox}{\stktypebox}
  \begin{bytefield}{32}
    \bitheader[endianness=big]{0-31}          \\
    \bitbox{6}{\stktypeclass}
    \bitbox{5}{#2}
    \bitbox{16}{0}
    \bitbox{5}{#1}
  \end{bytefield}
  \end{lrbox}
}


% \popstkstype{mne}{opc}{semantics}{description}
\newcommand{\popstktype}[4]{
  \begin{InstructionInfo}{Instruction Data}
  \item[Encoding] \stkbox{#2}{\dr} \usebox{\stktypebox}
  \item[Semantics]   #3
  \item[Syntax] \begin{alltt} #1  \dr \end{alltt}
  \end{InstructionInfo}

  #4
}


% \pushstkstype{mne}{opc}{semantics}{description}
\newcommand{\pushstktype}[4]{
  \begin{InstructionInfo}{Instruction Data}
  \item[Encoding] \stkbox{#2}{\sr{0}} \usebox{\stktypebox}
  \item[Semantics]   #3
  \item[Syntax] \begin{alltt} #1  \dr \end{alltt}
  \end{InstructionInfo}

  #4
}


\newsavebox{\stkframetypebox}
\newcommand{\stkframebox}[2]{
  \begin{lrbox}{\stkframetypebox}
  \begin{bytefield}{32}
    \bitheader[endianness=big]{0-31}          \\
    \bitbox{6}{\stktypeclass}
    \bitbox{5}{\dr}
    \bitbox{16}{#2}
    \bitbox{5}{#1}
  \end{bytefield}
  \end{lrbox}
}

\newcommand{\stkframetype}[5]{
  \begin{InstructionInfo}{Instruction Data}
  \item[Encoding] \stkframebox{#3}{#2} \usebox{\stkframetypebox}
  \item[Semantics]   #4
  \item[Syntax] \begin{alltt} #1 \dr, \texttt{words} \end{alltt}
  \end{InstructionInfo}

  #5
}


\newsavebox{\bttypebox}
\newcommand{\btbox}[2]{
  \begin{lrbox}{\bttypebox}
  \begin{bytefield}{32}
    \bitheader[endianness=big]{0-31}          \\
    \bitbox{6}{\bttypeclass}
    \bitbox{5}{\dr}
    \bitbox{5}{#1}
    \bitbox{5}{\sr{1}}
    \bitbox{6}{0}
    \bitbox{5}{#2}
  \end{bytefield}
  \end{lrbox}
}

\newcommand{\btrrtype}[4]{
  \begin{InstructionInfo}{Instruction Data}
  \item[Encoding] \btbox{\sr{0}}{#2} \usebox{\bttypebox}
  \item[Semantics]   #3
  \item[Syntax] \begin{alltt} #1 \sr{0}, \sr{1}, \dr \end{alltt}
  \end{InstructionInfo}

  #4
}

\newcommand{\btcrtype}[4]{
  \begin{InstructionInfo}{Instruction Data}
  \item[Encoding] \btbox{C}{#2} \usebox{\bttypebox}
  \item[Semantics]   #3
  \item[Syntax] \begin{alltt} #1 \textrm{C}, \sr{1}, \dr \end{alltt}
  \end{InstructionInfo}

  #4
}

% Integer Register Instructions
% \rtype{mnemonic}{opcode}{mathematical description}{semantics}
\newsavebox{\rtypebox}
\newcommand{\rbox}[1]{
  \begin{lrbox}{\rtypebox}
  \begin{bytefield}{32}
    \bitheader[endianness=big]{0-31}          \\
    \bitbox{6}{\rtypeclass}
    \bitbox{5}{\dr}
    \bitbox{5}{\sr{0}}
    \bitbox{5}{\sr{1}}
    \bitbox{6}{0}
    \bitbox{5}{#1}
  \end{bytefield}
  \end{lrbox}
}

\newcommand{\rtype}[4]{
  \begin{InstructionInfo}{Instruction Data}
  \item[Encoding] \rbox{#2}\usebox{\rtypebox}
  \item[Semantics]   #3
  \item[Syntax] \begin{alltt} #1 \sr{0}, \sr{1}, \dr \end{alltt}
  \end{InstructionInfo}

  #4
}

% \gregbox{mne}{sr0 bank}{sr1 bank}{dr bank}
\newsavebox{\gregtypebox}
\newcommand{\gregbox}[4]{
  \begin{lrbox}{\gregtypebox}
    \begin{bytefield}{32}
      \bitheader[endianness=big]{0-31}          \\
      \bitbox{6}{\gregtypeclass}
      \bitbox{5}{\dr}
      \bitbox{5}{\sr{0}}
      \bitbox{5}{\sr{1}}
      \bitbox{1}{\regbank{#4}}
      \bitbox{1}{\regbank{#3}}
      \bitbox{1}{\regbank{#2}}
      \bitbox{3}{0}
      \bitbox{5}{#1}
    \end{bytefield}
  \end{lrbox}
}

% Arithmetic Instructions
% \gregtype{mnemonic}{opcode}{mathematical description}{semantics}
%
\newcommand{\gregtype}[4]{
  \begin{InstructionInfo}{Instruction Data}
  \item[Encoding] \gregbox{#2}{b}{b}{b} \usebox{\gregtypebox}
  \item[Semantics] #3
  \item[Syntax] \begin{alltt} #1 \sr{0}, \sr{1}, \dr \end{alltt}
  \end{InstructionInfo}

  #4
}

% \absbox{abs-opc}
% abs instruction (which has special requirements on the output register)
\newcommand{\absbox}[1]{
  \begin{lrbox}{\gregtypebox}
    \begin{bytefield}{32}
      \bitheader[endianness=big]{0-31}          \\
      \bitbox{6}{\gregtypeclass}
      \bitbox{5}{0}
      \bitbox{5}{\sr{0}}
      \bitbox{5}{\sr{1}}
      \bitbox{1}{0}
      \bitbox{1}{\regbank{0}}
      \bitbox{1}{\regbank{1}}
      \bitbox{3}{0}
      \bitbox{5}{#1}
    \end{bytefield}
  \end{lrbox}
}

\newcommand{\abstype}[4]{
  \begin{InstructionInfo}{Instruction Data}
  \item[Encoding] \absbox{#2} \usebox{\gregtypebox}
  \item[Semantics] #3
  \item[Syntax] \begin{alltt} #1 \sr{0}, \sr{1}, \dr \end{alltt}
  \end{InstructionInfo}

  #4
}


% \cmpbox{cmp-opc}
% cmp instruction (which has special requirements on the output register)
\newcommand{\cmpbox}[1]{
  \begin{lrbox}{\gregtypebox}
    \begin{bytefield}{32}
      \bitheader[endianness=big]{0-31}          \\
      \bitbox{6}{\gregtypeclass}
      \bitbox{5}{0}
      \bitbox{5}{\sr{0}}
      \bitbox{5}{\sr{1}}
      \bitbox{1}{0}
      \bitbox{1}{\regbank{0}}
      \bitbox{1}{\regbank{1}}
      \bitbox{3}{0}
      \bitbox{5}{#1}
    \end{bytefield}
  \end{lrbox}
}

\newcommand{\cmptype}[4]{
  \begin{InstructionInfo}{Instruction Data}
  \item[Encoding] \cmpbox{#2} \usebox{\gregtypebox}
  \item[Semantics] #3
  \item[Syntax] \begin{alltt} #1 \sr{0}, \sr{1}, \dr \end{alltt}
  \end{InstructionInfo}

  #4
}


% Conditional Set Instructions
\newsavebox{\ctypebox}
\newcommand{\cbox}[1]{
  \begin{lrbox}{\ctypebox}
    \begin{bytefield}{32}
      \bitheader[endianness=big]{0-31}          \\
      \bitbox{6}{\ctypeclass}
      \bitbox{5}{\dr}
      \bitbox{5}{\sr{0}}
      \bitbox{11}{0}
      \bitbox{5}{#1}
    \end{bytefield}
  \end{lrbox}
}

% \ctype{mnemonic}{opcode}{semantics}
\newcommand{\ctype}[3]{
  \begin{InstructionInfo}{Instruction Data}
  \item[Encoding] \cbox{#2}\usebox{\ctypebox}
  \item[Semantics]   #3
    \\
  \item[Syntax] \begin{alltt} #1 \sr{0}, \dr \end{alltt}
  \end{InstructionInfo}

  Sets \dr to the expression in the semantics section using the
  \xrefinst{cmp} result in \sr{0}.
}


% System Trap Instructions
% \trapbox{25..21}{15..08}{07..00}
\newsavebox{\traptypebox}
\newcommand{\trapbox}[3]{
  \begin{lrbox}{\traptypebox}
    \begin{bytefield}{32}
      \bitheader[endianness=big]{0-31}          \\
      \bitbox{6}{\systraptypeclass}
      \bitbox{5}{#1}
      \bitbox{5}{\sr{0}}
      \bitbox{8}{#2}
      \bitbox{8}{#3}
    \end{bytefield}
  \end{lrbox}
}

%% TODO: trapnil does not have a second argument.  Need special
%% macro for trapnil.
% \traptype{mnemonic}{trapcode}{opcode}{semantics}{description}
\newcommand{\systraptype}[5]{
  \begin{InstructionInfo}{Instruction Data}
  \item[Encoding] \trapbox{\sr{1}}{#2}{#3}\usebox{\traptypebox}
  \item[Semantics]   #4
  \item[Syntax] \begin{alltt} #1 \sr{0}, #3 \end{alltt}
  \end{InstructionInfo}

  #5
}


\newcommand{\niltraptype}[5]{
  \begin{InstructionInfo}{Instruction Data}
  \item[Encoding] \trapbox{0}{#2}{#3}\usebox{\traptypebox}
  \item[Semantics]   #4
  \item[Syntax] \begin{alltt} #1 \sr{0} \end{alltt}
  \end{InstructionInfo}

  #5
}


\newcommand{\rangetraptype}[5]{
  \begin{InstructionInfo}{Instruction Data}
  \item[Encoding] \trapbox{0}{#2}{#3}\usebox{\traptypebox}
  \item[Semantics]   #4
  \item[Syntax] \begin{alltt} #1 \sr{0}, #3 \end{alltt}
  \end{InstructionInfo}

  #5
}

% Sign and Zero extend Instructions
% \exttype{mnemonic}{opcode-extension}{description}
\newsavebox{\exttypebox}
\newcommand{\extbox}[1]{
  \begin{lrbox}{\exttypebox}
  \begin{bytefield}{32}
  \bitheader[endianness=big]{0-31}          \\
  \bitbox{6}{\exttypeclass}
  \bitbox{5}{\dr}
  \bitbox{5}{\sr{0}}
  \bitbox{11}{0}
  \bitbox{5}{#1}
  \end{bytefield}
  \end{lrbox}
}

\newcommand{\exttype}[4]{       % {math-expression}{opcode}{semantics}{description}
  \begin{InstructionInfo}{Instruction Data}
  \item[Encoding] \extbox{#2}\usebox{\exttypebox}
  \item[Semantics]   #3
  \item[Syntax] \begin{alltt} #1 \sr{0}, \dr \end{alltt}
  \end{InstructionInfo}

  #4
}


% Load Register Instructions
%
\newsavebox{\regmemtypebox}
\newcommand{\regmembox}[1]{
  \begin{lrbox}{\regmemtypebox}
    \begin{bytefield}{32}
      \bitheader[endianness=big]{0-31} \\
      \begin{leftwordgroup}{0}
        \bitbox{6}{\regmemtypeclass}
        \bitbox{5}{\dr}
        \bitbox{5}{\sr{base}}
        \bitbox{5}{\sr{index}}
        \bitbox{3}{0}
        \bitbox{2}{S}
        \bitbox{1}{0}
        \bitbox{5}{#1}
      \end{leftwordgroup} \\
      \begin{leftwordgroup}{4}
        \bitbox{32}{signed offset}
      \end{leftwordgroup}
    \end{bytefield}
  \end{lrbox}
}

% \regmemtype{mnemonic}{opcode}{semantics}{description}
%
\newcommand{\regmemtype}[4]{
  \begin{InstructionInfo}{Instruction Data}
  \item[Encoding] \regmembox{#2}\usebox{\regmemtypebox}
  \item[Semantics]
      #3

  \item[Syntax]
    \begin{alltt} #1 \sr{base}, \sr{index}, S, offset,  \dr \end{alltt}
  \end{InstructionInfo}

  #4

  Refer to \xref{sect:register-memory}.
}


% \laregmemtype{mnemonic}{opcode}{description}
%
%  Slight variation of \regmemtype, because 'la' does not access
%  memory.
%
\newcommand{\laregmemtype}[3]{
  \begin{InstructionInfo}{Instruction Data}
  \item[Encoding] \regmembox{#2}\usebox{\regmemtypebox}
  \item[Semantics]
    \begin{math}
      \begin{array}{lcl}
        \dr & \becomes & \textrm{\sr{base}} + (\textrm{\sr{index}}
                         \times 2^\texttt{S}) + \texttt{signed offset} \\
      \end{array}
    \end{math}

  \item[Syntax]
    \begin{alltt} #1 \sr{base}, \sr{index}, S, offset,  \dr \end{alltt}
  \end{InstructionInfo}

  #3

  Refer to \xref{sect:register-memory}.
}


% Load Register Immediate Instructions
% \regmemimmtype{mnemonic}{opcode}{description}

\newcommand{\regmemimmbox}[1]{
  \begin{lrbox}{\regmemtypebox}
    \begin{bytefield}{32}
      \bitheader[endianness=big]{0-31} \\
      \begin{leftwordgroup}{0}
        \bitbox{6}{\regmemtypeclass}
        \bitbox{5}{\dr}
        \bitbox{5}{\sr{base}}
        \bitbox{11}{0}
        \bitbox{5}{#1}
      \end{leftwordgroup} \\
      \begin{leftwordgroup}{4}
        \bitbox{32}{constant}
      \end{leftwordgroup}
    \end{bytefield}
  \end{lrbox}
}

\newcommand{\regmemimmtype}[3]{
  \begin{InstructionInfo}{Instruction Data}
  \item[Encoding] \regmemimmbox{#2}\usebox{\regmemtypebox}
  \item[Semantics] $\dr \becomes \texttt{\sr{base}} + \texttt{constant}$
  \item[Syntax] \begin{alltt} #1 \texttt{constant}(\sr{base}), \dr \end{alltt}
  \end{InstructionInfo}

  #3
}

\newsavebox{\regmemimmfloattypebox}
\newcommand{\regmemimmfloatbox}[1]{
  \begin{lrbox}{\regmemimmfloattypebox}
    \begin{bytefield}{32}
      \bitheader[endianness=big]{0-31} \\
      \begin{leftwordgroup}{0}
        \bitbox{6}{\regmemtypeclass}
        \bitbox{5}{\dr}
        \bitbox{5}{0}
        \bitbox{11}{0}
        \bitbox{5}{#1}
      \end{leftwordgroup} \\
      \begin{leftwordgroup}{4}
        \bitbox{32}{constant}
      \end{leftwordgroup}
    \end{bytefield}
  \end{lrbox}
}

\newcommand{\regmemimmfloattype}[3]{
  \begin{InstructionInfo}{Instruction Data}
  \item[Encoding] \regmemimmfloatbox{#2}\usebox{\regmemimmfloattypebox}
  \item[Semantics] $\dr \becomes \texttt{constant}$
  \item[Syntax] \begin{alltt} #1 \texttt{constant}, \dr \end{alltt}
  \end{InstructionInfo}

  #3
}


\newsavebox{\regmemimmdoubletypebox}
\newcommand{\regmemimmdoublebox}[1]{
  \begin{lrbox}{\regmemimmdoubletypebox}
    \begin{bytefield}{32}
      \bitheader[endianness=big]{0-31} \\
      \begin{leftwordgroup}{0}
        \bitbox{6}{\regmemtypeclass}
        \bitbox{5}{\dr}
        \bitbox{5}{0}
        \bitbox{11}{0}
        \bitbox{5}{#1}
      \end{leftwordgroup} \\
      \begin{leftwordgroup}{4}
        \bitbox{32}{lo}
      \end{leftwordgroup} \\
      \begin{leftwordgroup}{8}
        \bitbox{32}{hi}
      \end{leftwordgroup}
    \end{bytefield}
  \end{lrbox}
}

\newcommand{\regmemimmdoubletype}[3]{
  \begin{InstructionInfo}{Instruction Data}
  \item[Encoding] \regmemimmdoublebox{#2}\usebox{\regmemimmdoubletypebox}
  \item[Semantics]
    \begin{math}
      \begin{array}{lcl}
        {\dr}_{63..32} & \becomes & \texttt{hi} \\
        {\dr}_{31..0} & \becomes & \texttt{lo} \\
      \end{array}
      \end{math}
  \item[Syntax] \begin{alltt} #1 \texttt{constant}, \dr \end{alltt}
  \end{InstructionInfo}

  #3
}


% Store Register Instructions
% \sregmemtype{mnemonic}{opcode}{semantics}{description}

\newcommand{\sregmemtype}[4]{
  \begin{lrbox}{\regmemtypebox}
    \begin{bytefield}{32}
      \bitheader[endianness=big]{0-31} \\
      \begin{leftwordgroup}{0}
        \bitbox{6}{\regmemtypeclass}
        \bitbox{5}{\sr{0}}
        \bitbox{5}{\sr{base}}
        \bitbox{5}{\sr{index}}
        \bitbox{3}{0}
        \bitbox{2}{S}
        \bitbox{1}{0}
        \bitbox{5}{#2}
      \end{leftwordgroup} \\
      \begin{leftwordgroup}{4}
        \bitbox{32}{signed offset}
      \end{leftwordgroup}
    \end{bytefield}
  \end{lrbox}

  \begin{InstructionInfo}{Instruction Data}
  \item[Encoding] \usebox{\regmemtypebox}
  \item[Semantics] #3
  \item[Syntax] \begin{alltt} #1 \sr{0}, \sr{base}, \sr{index}, S, offset\end{alltt}
  \end{InstructionInfo}

  #4
}


% Control register instructions (load from control register)
% \lcrtype{mnemonic}{description}
\newsavebox{\lcrtypebox}
\newcommand{\lcrbox}{
  \begin{lrbox}{\lcrtypebox}
  \begin{bytefield}{32}
  \bitheader[endianness=big]{0-31}          \\
  \bitbox{6}{\crtypeclass}
  \bitbox{5}{\dr}
  \bitbox{5}{\acs{CR}}
  \bitbox{11}{0}
  \bitbox{5}{\crlcr} \\
  \end{bytefield}
  \end{lrbox}
}

\newcommand{\lcrtype}[2]{
  \begin{InstructionInfo}{Instruction Data}
  \item[Encoding] \lcrbox\usebox{\lcrtypebox}
  \item[Semantics]   $\dr \becomes \textrm{\acs{CR}}$
  \item[Syntax] \begin{alltt} #1 \acs{CR}, \dr \end{alltt}
  \end{InstructionInfo}

  #2
}

% Control register instructions (store to control register)
% \scrtype{mnemonic}{description}
\newsavebox{\scrtypebox}
\newcommand{\scrbox}{
  \begin{lrbox}{\scrtypebox}
  \begin{bytefield}{32}
  \bitheader[endianness=big]{0-31}          \\
  \bitbox{6}{\crtypeclass}
  \bitbox{5}{\acs{CR}}
  \bitbox{5}{\sr{0}}
  \bitbox{11}{0}
  \bitbox{5}{\crscr}
  \end{bytefield}
  \end{lrbox}
}

\newcommand{\scrtype}[2]{
  \begin{InstructionInfo}{Instruction Data}
  \item[Encoding] \scrbox\usebox{\scrtypebox}
  \item[Semantics]   $\textrm{\acs{CR}} \becomes \sr{0}$
  \item[Syntax] \begin{alltt} #1 \dr, \acs{CR} \end{alltt}
  \end{InstructionInfo}

  #2
}


% Unconditional Jump to register & link instructions
% \jraltype{description}
\newsavebox{\jraltypebox}
\newcommand{\jralbox}{
  \begin{lrbox}{\jraltypebox}
    \begin{bytefield}{32}
      \bitheader[endianness=big]{0-31}          \\
      \bitbox{6}{\jraltypeclass}
      \bitbox{5}{\dr}
      \bitbox{5}{\sr{0}}
      \bitbox{16}{0}
    \end{bytefield}
  \end{lrbox}
}

\newcommand{\jraltype}[1]{
  \begin{InstructionInfo}{Instruction Data}
  \item[Encoding] \jralbox\usebox{\jraltypebox}
  \item[Semantics]
    \begin{math}
      \begin{array}{lcl}
        \dr & \becomes & \pc + 4 \\
        \pc & \becomes & \sr{0} \\
      \end{array}
    \end{math}
  \item[Syntax] \begin{alltt} jral \sr{0}, \dr \end{alltt}
  \end{InstructionInfo}

  #1
}


% Jump Instructions
% \jumptype{20..16}{4..0}
%          {reg}{opcode}   first arg always 0? Remove!
\newsavebox{\jumptypebox}
\newcommand{\jumpbox}[2]{
  \begin{lrbox}{\jumptypebox}
    \begin{bytefield}{32}
      \bitheader[endianness=big]{0-31} \\
      \begin{leftwordgroup}{0}
        \bitbox{6}{\jumptypeclass}
        \bitbox{5}{0}
        \bitbox{5}{#1}
        \bitbox{11}{0}
        \bitbox{5}{#2}
      \end{leftwordgroup} \\
      \begin{leftwordgroup}{4}
        \bitbox{32}{unsigned relative address}
      \end{leftwordgroup}
    \end{bytefield}
  \end{lrbox}
}

\newcommand{\jjumptype}[5]{    % Unconditional jump.
  \begin{InstructionInfo}{Instruction Data}
  \item[Encoding] \jumpbox{#3}{#4}\usebox{\jumptypebox}
  \item[Semantics]
    $\pc \becomes \pc + 8 + \texttt{address}$

  \item[Syntax] \begin{alltt} #1 address \end{alltt}
  \end{InstructionInfo}

  #5
}


\newcommand{\jaljumptype}[5]{    % Unconditional jump.
  \begin{InstructionInfo}{Instruction Data}
  \item[Encoding] \jumpbox{#3}{#4}\usebox{\jumptypebox}
  \item[Semantics]
    \begin{math}
      \begin{array}{lcl}
          \pc & \becomes & \pc + 8 + \texttt{address} \\
          \sr{31} & \becomes & \pc + 8 \\
      \end{array}
    \end{math}
  \item[Syntax] \begin{alltt} #1 address \end{alltt}
  \end{InstructionInfo}

  #5
}


% Conditional jumps.
% #1 { mnemonic }
% #3 { opcode }
% #4 { Description }
% #5 { condition (math) }
\newcommand{\jumptype}[4]{
  \begin{InstructionInfo}{Instruction Data}
  \item[Encoding] \jumpbox{\sr{0}}{#2}\usebox{\jumptypebox}
  \item[Semantics]
    \begin{math}
      \pc \becomes \left\lbrace
      \begin{array}{rcl}
        \neg (#4) & \implies & \pc + 8  \\
        (#4)      & \implies & \pc + 8 + \texttt{address}  \\
      \end{array}
      \right.
    \end{math}

    The value contained in \sr{0} must be of the same format as
    produced by \xrefinst{cmp}.

  \item[Syntax] \begin{alltt} #1 address \end{alltt}
  \end{InstructionInfo}

  #3
}

% Miscellanous Instructions
\newsavebox{\mtypebox}
\newcommand{\miscbox}[3]{
  \begin{lrbox}{\mtypebox}
    \begin{bytefield}{32}
      \bitheader[endianness=big]{0-31} \\
      \bitbox{6}{\mtypeclass}
      \bitbox{5}{0}
      \bitbox{5}{#2}
      \bitbox{5}{#3}
      \bitbox{6}{0}
      \bitbox{5}{#1} \\
    \end{bytefield}
  \end{lrbox}
}

% \mtype{mnemonic}{opcode}{20..16}{15..11}{semantics}{description}{asm operands}
\newcommand{\mtype}[7]{
  \begin{InstructionInfo}{Instruction Data}
  \item[Encoding] \miscbox{#2}{#3}{#4}\usebox{\mtypebox}
  \item[Semantics] #5
  \item[Syntax] \begin{alltt} #1 #7 \end{alltt}
  \end{InstructionInfo}

#6
}

% Special store to register Instructions
% \sregtype{mnemonic}{opcode}{semantics}{description}{assembly-operands}
\newsavebox{\sregtypebox}
\newcommand{\sregbox}[1]{
  \begin{lrbox}{\sregtypebox}
    \begin{bytefield}{32}
      \bitheader[endianness=big]{0-31} \\
      \bitbox{6}{\sregtypeclass}
      \bitbox{5}{\instreg{d}}
      \bitbox{16}{0}
      \bitbox{5}{#1} \\
    \end{bytefield}
  \end{lrbox}
}

\newcommand{\sregtype}[5]{
  \begin{InstructionInfo}{Instruction Data}
  \item[Encoding] \sregbox{#2}\usebox{\sregtypebox}
  \item[Semantics] #3
  \item[Syntax] \begin{alltt} #1 \dr, #5\end{alltt}
  \end{InstructionInfo}

  #4
}


% From bytefield.dtx documentation.
%
\newcommand{\memsection}[4]{%
  % define the height of the memsection
  \bytefieldsetup{bitheight=#3\baselineskip}%
  \bitbox[]{10}{%
    \texttt{#1}%      print end address
    \\
    %   do some spacing
    \vspace{#3\baselineskip}
    \vspace{-2\baselineskip}
    \vspace{-#3pt}
    \texttt{#2}%      print start address
  }%
  \bitbox{16}{#4}%    print box with caption
}

\newcommand{\lchevron}{\ensuremath{\ll}}
\newcommand{\rchevron}{\ensuremath{\gg}}

%%% Local Variables:
%%% mode: latex
%%% TeX-master: "cpu"
%%% End:
