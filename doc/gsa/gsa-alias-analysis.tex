% Copyright (c) 2001-2022 Logic Magicians Software
\chapter{Alias Analysis}

\ac{aa} is a process to determine if access to memory through two or
more different data paths yields the same memory location.  For a
compiler which does not invoke any \ac{cia}, this is generally not a
problem since all access to data will directly access memory.
However, for a compiler which does perform \ac{cia}, it may turn out
that the value of an aliased memory be held in a register - in which
case access to the physical memory can yield the wrong data.

Consider, for example:

\begin{verbatim}
PROCEDURE P0(VAR x, y : INTEGER);
BEGIN
  y := 10;
  INC(x);
  IF y = 10 THEN x := 0;
  ELSE x := 1;
  END;
END P0;

PROCEDURE P1;
  VAR z : INTEGER;
BEGIN z := 0; P0(z, z); ASSERT(z = 1);
END P1;
\end{verbatim}

A non-\ac{ci} compiler would produce correct code because all access
to both \code{x} and \code{y} would be directly to memory.  However,
if a \ac{ci} compiler, without \ac{aa}, placed \code{y} into a
register then the output value of \code{x} will be the, incorrect,
value \code{0}.

\section{Definition of Terms}
\begin{description}
\item[alias] An \emph{alias} occurs when two accesses to memory definitely
  refer to the same memory location.
\item[may alias] A \emph{may alias} occurs when two memory references
  \emph{might} refer to the same memory location.
\item[no alias] A \emph{no alias} indicates that two memory references
  definitely do not refer to the same memory location.
\end{description}

\section{Definition of Work}

\ac{aa} must ensure that access to memory always yields the current
value of that memory.

\section{Description of Work}


