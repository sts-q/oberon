% Copyright (c) 2001-2022 Logic Magicians Software
\chapter{Notes}

\section{Initialization of Local Variables}
The translation to gsa will implicitly set all local variables to the
value binary 0.  This can cause the generated code to be different
than expected, if those variables are, in fact, not initialized by the
program source due to dce.

Consider:

\begin{verbatim}
b : BOOLEAN;

IF b THEN PerformSomethingAmazing(); END;
\end{verbatim}

The initialization also is done in the module initialization code for
module variables.

\section{Assginment to Procedure Variable}
When taking the address of a global procedure for assignment to a
procedure variable, the AST should explicitly take the address of the
procedure.  This make conversion to gsa a bit easier since it removes
a special case of procedure variable assignments (the difference
between assigning a procedure variable to another procedure variable
-vs- assigning a procedure to procedure variable).  This also impacts
the code generation for assignments from the AST - which is a
simplification as well.

A good solution to this problem is to make the AST more explicit: when
assigning a global procedure to a procedure variable, insert a 'take
address' node.  This removes any special case in any type of backend
which will traverse the tree.

I'm ambivilent about doing the same for by-reference parameters, however.

\section{Open Array}
The syntax tree should explicitly indicate the array length parameter
to simplify the backend - especially O3MGSA.Enter: no special case
would be needed for processing parameters (to add open array lengths
and record descriptors) since they would be explicit in the AST.

\section{indirect instruction}

I think an instruction for providing access to indirect memory is
necessary, and could possibly simplify some constructions into lower
level items.  For example, \$tbpadr could be turned into several
lower-level instructions which model more closely any target cpu.

\%mem := indirect.asU4  \%mem, base-address, offset-in-bytes

As a start, this would load an unsigned 4 byte value indirectly from
'base-address + offset-in-bytes'.

A model for writing memory in the same fashion would be needed.

\section{AST generation}

It may be useful to have a flag set on \emph{compiler generated} nodes
in the AST.  For example, the Oberon compiler generates halt
nodes when the ELSE clasue is not present on CASE and WITH
statements... to be able to distinguish such compiler-generated code
can be useful: color coding in a debugger, code generation phases
could use this information to disable compiler generated code.  Or,
different code sequences can be emitted (translating a hard halt call
into a \$trap instruction, for example).

\subsection{Standard Procedures}
Standard procedures should be mapped into function calls if a result
is returned.  For example, \code{NEW(p)} should become \code{p :=
  new-record(type-descriptor)}.  This allows the conversion to the
\ac{gsa} form can easily leverage all the other infrastructure to
handle the assignment.

