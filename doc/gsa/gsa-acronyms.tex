% Copyright (c) 2001-2022 Logic Magicians Software
\chapter{Acronyms}

\begin{acronym}
  \acro{ci}[CI]{Code Improvement}

  \acro{cia}[CIA]{Code Improvement Algorithm}

  A \acs{cia} is most commonly called an \emph{optimization} by many
  other compiler textbooks and papers, however it is actually a
  misnomber because no \emph{optimization} algorithm will actually
  produce the optimal code for any given input.

  Instead, this document will use the more correct term \ac{cia}.
  
  \acro{ast}[AST]{Abstract Symbol Tree}
  \acro{gsa}[GSA]{Guarded Static Assignment}
  \acro{ir}[IR]{Intermediate Representation}
  \acro{mr}[MR]{Machine Representation}
  
  This intermediate form is a direct represention of the instructions
  which the target machine supports.  It is only amenable to peephole
  optimizations.

  \acro{tbp}[TBP]{Type Bound Procedure}
  
  \acro{mt}[MT]{Method Table}
  
  \acro{td}[TD]{Type Descriptor}
  \acro{fe}[FE]{Compiler Frontend}
  \acro{be}[BE]{Compiler Backend}

  \acro{lv}[lvalue]{left-hand-side value}
  
  An \emph{lvalue} is an expression which appears on the
  left-hand-side of an assignment statement.  It must evaluate,
  ultimately, to a memory location, rather than a value.

  \acro{rv}[rvalue]{right-hand-side value}
  An \emph{rvalue} is any expression which does not appear on the
  left-hand-side of an assignment statement, or cannot be evaluated to
  a memory address.  For example, \code{SYSTEM.ADR(v) := 10;} is an
  illegal statement because \code{SYSTEM.ADR} returns a \code{LONGINT}
  value, and a value cannot appear on th left-hand-side of an
  assignment.

  \acro{cse}[CSE]{Common Subexpression Elimination}
  The compiler contains algorithms which will determine if two
  calculations are congruent\footnote{produce the same value}.  Two or
  more values which compute the same value are called \emph{common
    subexpressions}; the compiler also contains algorithms which
  remove such common subexpressions that are executed under the same
  control path.

  \acro{greg}[greg]{Global Region}

  \acro{rmld}[rmld]{Runtime Memory Layout Document}

  This document describes the layout of process memory.

  \acro{olid}[olid]{Oberon Language Implementation Document}

  This document describes the method used to implement an Oberon compiler.

  \acro{phi}[$\phi$]{$\phi$}
\marginpar{A better explanation here would be nice.}  
  In \ac{gsa}, a \emph{gate}.  A method of assigning a single value to
  a variable which is set to different values in different control
  paths.  Consider:

\begin{verbatim}
  IF k < 10 THEN q := 151;
  ELSE q := 191;
  END;
\end{verbatim}

  When translated to \ac{gsa}, there will be a \emph{gate} following
  the \code{IF} which sets \code{q} to either \code{151} or
  \code{191} - depending on the control path taken through the
  executable code.  See \S\ref{class:merge}.

  \acro{dce}[DCE]{Dead Code Elimination}

  \acro{cp}[CP] {Copy Propagation}

  \acro{ccp}[CCP] {Conditional Constant Propagation \& Unreachable
  Code Elimination}
  
  This is a \ac{cia} in which copies of values are propogated
  throughout the \ac{ir}.  The result is a direct use of the original
  value rather than a copy of it.

  \acro{gc}[GC]{Garbage Collection}

  \acro{aa}[AA]{Alias Analysis}

  \acro{vn}[VN]{Value Numbering}

  \acro{fpu}[FPU]{Floating Point Unit}
\end{acronym}
