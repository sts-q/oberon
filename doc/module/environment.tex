\chap{Environment}

This module provides access to the host OS' shell environment
variables.  Environment variables can be changed within the Oberon
environment, but any change made will not be reflected into the shell
environment from which the Oberon program was launched.  Moreover,
changes to the shell environment made after the Oberon program was
launched will not be reflected in Oberon.

These variables are key-value pairs specified in the shell
environment that can be used to configure runtime aspects of programs.
For example, a default directory for file output, or a search path for
program executables.

As a key-value pair, the shell environment variables have the
following form:

\begin{alltt}

  Key     := name
  Value   := element \{ ':' element \}

  Name    := letter \{ letter | digit | '_' | \}
  element := <ASCII characters except ':'>
\end{alltt}

\section{Source}

\begin{tabularx}{\textwidth}{lX}
  Source & \texttt{SKLEnvironment.Mod} \\
  Test & \texttt{CTEnvironment.Mod} \\
\end{tabularx}

\section{Constants}

There are no exported constants from this module.

\section{Types}
\subsection{\texttt{Text}}\label{environment:text}
\begin{alltt}
Text = POINTER TO ARRAY OF CHAR;
\end{alltt}

This type is used to refer to data that is stored in the internal
\emph{kev-value} cache.  In terms of this module exported function, it
refers to the raw \emph{value} associated with a key.

\subsection{\texttt{Elements}}\label{environment:elements}
\begin{alltt}
Elements = POINTER TO ARRAY OF Text;
\end{alltt}

This type is used to refer to individual values associated with a
\emph{key} in the internal \emph{key-value} cache.  When the raw data
(\xref{environment:text}) is split on a separator character, this type
is produced.


\section{Procedures}

\subsection{Delete}
\begin{alltt}
  PROCEDURE Delete*(key : ARRAY OF CHAR);
\end{alltt}

This procedure removes \emph{key} from the internal data cache.  If
\emph{key} does not exist in the cache, nothing happens.

The input argument, \emph{key}, must be ASCIIZ.

\subsection{Set}
\begin{alltt}
  PROCEDURE Set*(key   : ARRAY OF CHAR;
                 value : ARRAY OF CHAR);
\end{alltt}

This procedure sets \emph{key} to \emph{value}.  If \emph{key} does not
exist in the internal data cache, it is added.  If \emph{key} exists,
the value is set to \emph{value}.

Both input arguments, \emph{key} and \emph{value}, must be ASCIIZ.

\subsection{Lookup}
\begin{alltt}
  PROCEDURE Lookup*(key : ARRAY OF CHAR) : Text;
\end{alltt}

This procedure looks up \emph{key} in the internal data cache.  If
\emph{key} is present, its value is returned.  If \emph{key} is not
present, \texttt{NIL} is returned.

The input argument, \emph{key}, must be ASCIIZ.


\subsection{Split}
\begin{alltt}
  PROCEDURE Split*(v         : Text;
                   separator : CHAR) : Elements;
\end{alltt}

This procedure splits the input, \emph{v}, using the character
\emph{separator} as a value separator and returns the separated
values.

If \emph{v} is \texttt{NIL}, \texttt{NIL} is returned.
Otherwise, the length of the return value is dependent upon the number
of \emph{separator} characters present in the input.

The test module provides examples of how this function can be used.


